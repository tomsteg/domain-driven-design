\documentclass[11pt,a4paper]{scrartcl}
% a4paper bestimmt die Papiergröße als Din A4
% 11pt legt die Schriftgröße auf 11pt fest
% scartcl ist die KOMA Sckript Klasse für article; der größte Unterschied besteht darin, dass die Überschriften in einer serifenlosen Schrift dargestellt werden.
\usepackage{ngerman} % bestimmt die deutsche Sprache mit den neuen Rechtschribung als Grundlage für die Wörterbücher und die Metawörter
\usepackage{eurosym} % ermögicht mit \euro den Zugriff auf ein dem Zeichensatz angepasstes Eurosymbol
\usepackage{textcomp} % Zugriff auf zusätzliche Textsymbole, zum Beispiel \texteuro, \textcopyright, \textregistered, \texttrademark, \textdollar, \texteuro, \textperthousand
\usepackage{fontenc} % lädt die Textfontkodierung; dies wird benötigt um die utf Kodierung verwenden zu können
\usepackage[utf8]{inputenc} % ermöglicht die Verwendung von utf8-kodierten Quelltexten
\usepackage{alltt} % definiert die alltt-Umgebung, eine verbatim-Umgebung, in der die Zeichen \,{,} ihre Gültigkeit behalten
\usepackage{graphicx} % ermöglicht mit dem \includegraphics Befehl das Einbinden von externen Graphiken
\usepackage{paralist} % erweitert die enumerate-Umgebung um einen optionalen Parameter, der die Art der Nummerierung angiebt: a), A., i), 1), Beispiel i), ...
\usepackage{tabularx} % ermöglicht die Verwendung von variablen Breiten mit der Formatanweisung X statt p{Breite} in der tabular-Umgebung
\usepackage{float} % der Parameter H in den Positionsoptionen von Gleitobjekten erzwingt die Positionierung an der Stelle im Quelltext
\usepackage{multicol} % ermöglicht die Verwendung des Befehls twocolumn auch in minipage oder figure Umgebungen, beziehungsweise die Verwendung der Umgebung multicols mit den Parametern {Spalten}[Vortext][Abstand]
\usepackage{fancyhdr} % ermöglicht die Verwendung eigener Kolumnentitel (Kopf- und Fusszeilen)
\usepackage{hyperref} % verwandelt alle Querverweise in Hyperlinks; es sollte als letztes Paket eingebunden werden
\newtheorem{Def}{Definition}[section]
\setlength{\parskip}{12pt}

\begin{document}

\graphicspath{{./images/}} % gibt den Pfad zu den Graphikdateien an, die eingebunden werden; hier das Unterverzeichnis bilder
\DeclareGraphicsExtensions{.jpg,.png} % erlaubte Graphikobjekte

\title{Domain Driven Design\\ \small eine Einleitung in das Thema}
\author{Thomas Steglich}
\date{\today}

\maketitle

\begin{abstract}
	\noindent In dem Buch~\cite[Domain Driven Design(DDD)]{dddp} wird das Prizip DDD erläutert. Einige Konzepte für eine moderne Software-Architektur können (sollen) daraus übernommen oder abgeleitet werden.
\end{abstract}

\pagestyle{fancy} % Layout Stil für Kopf- und Fusszeilen
\lhead{Domain Driven Design} % linker,
\chead{} % mittlerer und
\rhead{\rightmark} % rechter Bereich der Kopfzeile
\lfoot{Thomas Steglich} % linker,
\cfoot{\thepage} % mittlerer und
\rfoot{\today} % rechter Bereich der Fusszeile, wobei thepage die Seitennummer ausgibt, und leftmark und rightmark die Kapitel und Unterkapitel ausgibt.
\renewcommand\headrulewidth{1pt} % gibt die Dicke der Linie zwischen Kopfzeile und Text an
\renewcommand\footrulewidth{1pt} % gibt die Dicke der Linie zwischen Text und Fusszeile an

\newpage % neue Seite für Inhaltsverzeichnis
\tableofcontents % Inhaltsverzeichnis
\newpage % neue Seite nach Inhaltsverzeichnis

\section{Warum DDD?}

\paragraph{Welche Probleme können durch den Einsatz von DDD vermieden werden?}
\begin{enumerate}
	\item Die Datenbank hat eine zu hohe Priorität im Vergleich zu Geschäftsprozessen und -vorgängen.
	\item Objekte und Operationen werden oft nicht den fachlichen Aufgaben entsprechend benannt. Das führt zu einer zu großen Lücke zwischen den Anwendern und den Entwicklern.
	\item Fachlogik wird von den Entwicklern in Komponenten installiert, die zur Benutzerschnittstelle oder zur Datenhaltung gehören. Außerdem werden Operationen zum Speichern in der Datenbank mitten in der Fachlogik ausgeführt.
	\item Defekte, langsame und sich gegenseitig behindernde Datenbankabfragen blockieren die Anwender.
	\item Die Entwickler schaffen falsche Abstraktionen um alle gegenwärtigen und möglichen zukünftigen Bedürfnissen gerecht zu werden.
	\item Services sind untereinander zu stark gekoppelt.
\end{enumerate}

\paragraph{Zitate}

\begin{quote}
	Die Alternative zu gutem Design ist schlechtes Design, nicht überhaupt kein Design.~\cite[Douglas Martin]{martin}
\end{quote}

\begin{quote}
	Die meisten Menschen machen den Fehler zu denken, dass es bei Design nur darum geht, wie es aussieht. \dots Es geht nicht nur darum, wie etwas aussieht und sich anfühlt. Design ist, wie etwas funktioniert. \cite[Steve Jobs]{}
\end{quote}

\section{Strategisches Design}

\begin{Def}[Domain]
	Im Rahmen von DDD versteht man unter der Domäne die Anwendungsdomäne, die fachliche Aufgabenstellung unter der die Software entwickelt wird.
	Steht die Anwendungsdomäne im Mittelpunkt und ist Basis für die Architektur der Software, dann spricht man von DDD.
\end{Def}

Im DDD geht es in erster Linie darum, eine Ubiquitous Language in einem bestimmten bounded context zu modellieren.

\begin{Def}[Bounded Context]
	Innerhalb eines bounded context hat jede Komponente eines Softwaremodells eine bestimmte Bedeutung und tut bestimmte Dinge. Die Komponenten innerhalb eines bounded context sind kontextspezifisch und haben ihre Bedeutung und Funktion im Einklang mit den anderen Komponenten im bounded context.
\end{Def}

\begin{Def}[Ubiquitous Language]
	Innerhalb eines bounded context gilt eine ubiquitous language. Die Begriffe innerhalb dieser Sprache sind eindeutig. Sie wird von den Teammitgliedern genutzt und ist Basis des Softwaremodells. Es muss gewährleistet sein, dass die Sprache jedem Teammitglied zur Verfügung steht.
\end{Def}

Bei der Entwicklung der ubiquitous language darf man sich nicht verzetteln. Letztlich besteht das Domänenmodell nicht aus Definitionen, Diagrammen oder Spezifikationen. Das Domänenmodell ist der Code und der Code ist das Modell (Reverse Engineering).

Jeweils einem Team wird ein bounded context zugewiesen. Für jeden bounded context sollte es ein von den anderen getrenntes Quellcode Repository geben. Für jeden bounded context sollte es ein Datenbank-Schema geben, das klar von den anderen abgegrenzt ist. Die Kunst besteht darin, diese Grenzen zu wahren und zu bestimmen. Die Gefahr besteht darin, dass der bounded context zu viel umfasst.

\paragraph{Was sind Hinweise darauf, dass ein bounded context aus dem Ruder läuft?}
\begin{itemize}
	\item Das Team wird zu groß. Die optimale Teamgröße ist erreicht, wenn das ganze Team von zwei großen Pizzas satt wird.
	\item Die Begriffe verschwimmen und bekommen mehrere Bedeutungen.
	\item Immer wieder neue ergänzende Konzepte rauben Ressourcen für die Pflege und Tests der bestehenden Komponenten.
	\item Die neuen Konzepte gehören nicht mehr zum Kern des bounded contexts.
	\item Man scheut die Abgrenzung zu einem neuen bounded context.
	\item Man möchte das Risiko eines neuen bounded contexts nicht offenlegen. Dabei ist das Risiko, wenn man nicht abgrenzt, viel größer.
\end{itemize}

Ein bounded context enthält aber nicht nur das Domänenmodell, sondern auch Komponenten, die sich um die Kommunikation mit der Außenwelt kümmern (Adaptor).

\begin{Def}[Subdomain]
	Meistens lässt sich die Geschäftsdomäne nicht mit einem bounded context darstellen. Er wäre zu groß. Daher unterteilt man in subdomains, die für sich wieder einen bounded context definieren.
\end{Def}

In Altsystemen ist meist DDD nicht (voll) umgesetzt. Oft entsprechen sie dem Pattern big ball of mud. Da in ihnen keine bounded context festgelegt wurden, spricht man auch von unbounded legacy systems. Man könnte diese Altsysteme in mehrere logische Domänenmodelle unterteilen und für jedes einzelne eine ubiquitous language definieren. Tatsächlich hilft dieses Vorgehen, wenn man auf DDD umsteigt oder das Altsystem in eine Umgebung von anderen DDD-Systemen integrieren will.

\begin{Def}[Context Mapping]
	An der Schnittstelle zwischen zwei Subdomains treffen zwei ubiquitous languages aufeinander. Die unterschiedlichen Begriffe und Operationen in den jeweiligen domains müssen auf einander gemappt werden. Dies bezeichnet man als context mapping.
\end{Def}

\paragraph{Arten von Mapping}
\begin{itemize}
	\item Partnership; die Ziele der beiden involvierten Teams sind voneinander abhängig; für die Synchronisierung nutzen sie Continous Integration;
	\item Shared Kernel; zwei Teams teilen sich ein kleines gemeinsames Modell als Schnittmenge von zwei bounded context; ein shared kernal erfordert ein hohes Maß an Kommunikation;
	\item Customer Supplier; der supplier (upstream) stellt zur Verfügung, was der customer (downstream) braucht;
	\item Conformist; der upstream stellt etwas zur Verfügung, was der downstream benutzen darf;
	\item Anticorruption Layer (ACL); das downstream Team erzeugt eine Übersetzungsschicht (ACL);
	\item Open Host Service; das upstream Team stellt einen offen angebotenen Dienst zur Verfügung; die bereit gestellte API sollte gut dokumentiert sein; Open Host Services sind leichter zu konsumieren als Supplier;
	\item Published Language; oft bietet ein Open Host Service eine Published Language an;
	\item Separate Ways; keine Verbindung zwischen zwei bounded context;
	\item Big Ball of Mud; Kein Kontext Mapping;
	\begin{itemize}
		\item ungewollte und unzulässige Beziehungen und Abhängigkeiten
		\item Eine Änderung schlägt Wellen über das ganze System.
		\item Nur über Generationen weitergegebenes Geheimwissen und einzelne Heldentaten --- alle Sprachen auf einmal sprechen --- retten das System vor dem endgültigen Kollaps.
	\end{itemize}
\end{itemize}

Wenn ein bounded context mit einem big ball of mud System ein context mapping eingeht, so sollte das neue bounded context System unbedingt eine ACL vorschalten, so dass es nicht die ubiquitous language des big ball of mud übernimmt.

\paragraph{Technische Lösungen für Context Mapping}
\begin{itemize}
	\item Remote Procedure Call (RPC) mit Simple Object Access Protocol (SOAP)
	\item RESTful HTTP
	\item Messaging
	\begin{itemize}
		\item ein client-bounded-context abonniert die domain events, die von ihrem eigenen oder einem anderen bounded context versendet werden;
		\item at least once delivery: durch eine Empfangsbestätigung durch das Messaging System ist gewährleistet, dass eine Nachricht ausgeliefert wurde;
		\item idempotent reciever: der Empfänger kann  damit umgehen, dass eine Nachricht mehrmals geschickt wurde, die Operation aber nur einmal ausgeführt werden soll;
		\item die versendeten Daten können angereichert oder nachgeladen werden;
	\end{itemize}
\end{itemize}

\section{Taktisches Design}

Beim taktischen Design werden Konzepte, die für die Architektur innerhalb eines bounded context angewendet werden können, entworfen.

\begin{Def}[aggregates]
	aggregates nehmen Kommandos entgegen oder reagieren auf Ereignisse.
	Sie überprüfen, ob diese zulässig sind.
	Anschließend werden sie verarbeitet. Dabei können neue Ereignisse und/oder Kommandos ausgelöst werden.
	Das aggregate darf Daten zu dem Geschäftsvorgang speichern. 
	Das aggregate muss vor und nach der Verarbeitung konsistent sein.
	Das bezieht sich aber nur auf das aggregate selbst und schließt gleichzeitig nicht andere aggregates mit ein.
	Der Zustand vor und nach der Verarbeitung muss reproduzierbar sein(unittest).
	aggregates sind Zusammenfassungen von entities und value objects und deren Assoziationen untereinander zu einer gemeinsamen transaktionalen Einheit.
	Von außen kann nur mit genau einer entity auf das aggregate zugegriffen werden.
	Auf die inneren entities darf nicht direkt zugegriffen werden.
\end{Def}

\begin{Def}[Entities, reference objects]
	Objekte, welche nicht durch ihre Eigenschaften, sondern durch ihre Identität definiert werden. Beispiel: Person. Die Objekte werden mit eindeutigen Identifikatoren modelliert.
\end{Def}

\begin{Def}[value objects]
	Objekte, die durch ihre Eigenschaft definiert werden; immutable objects, wiederverwendbar,
\end{Def}

\paragraph{Regeln für aggregates}
\begin{enumerate}
	\item\label{aggrulekons} Schütze die fachliche Konsistenz von aggregates. Nach einem commit einer transaction müssen die Daten konsistent sein.
	\item Entwirf kleine aggregates, dann ist die Einhaltung der Regel~\ref{aggrulekons} leichter. Single Responsibility Principle~(SRP).
	\item Referenziere andere aggregates nur über ihre Identität.
	\item Aktualisiere andere aggregates unter Verwendung von eventual consistency.
	\item Wähle das Abstraktions-Level mit Bedacht. Ein zu hohes Abstraktions-Level führt zu vielen Spezialfällen, zu zu viel code. Man wird nie alle zukünftigen Bedürfnisse abdecken können.
	\item Sichere aggregates durch unittests ab.
	\item Ausgangsbasis für die Modellierung von aggregates kann ein Aktionsdiagramm sein, in welchem die Ereignisse in der Domäne zusammengefasst werden.
\end{enumerate}

\section{Fazit}

Im Kern von DDD geht es darum, geschäftliche Zusammenhänge exakt zu analysieren, eine passende Sprache zu entwickeln und mit ihrer Hilfe die Fachlichkeit in Programmcode abzubilden.

\nocite{*} % alle Literaturangaben werden genannt, auch wenn keine Verweise auf sie zeigen
\bibliographystyle{plain}
\begin{thebibliography}{\hspace{1cm}}
	\bibitem [1]{dddp} ``Domain Driven Design'' von Vaughn Vernon, aus dem Englischen übersetzt von Carola Lilienthal und Henning Schwertner
	\bibitem [2]{martin} ``Book Design: A Practical Introduction'' von Douglas Martin, 1990
	\bibitem [3]{wikipedia} Wikipedia: Domain Driven Design \url{https://de.wikipedia.org/wiki/Domain-driven_Design}
	\bibitem [4]{heise} \url{https://www.heise.de/developer/artikel/Domain-driven-Design-erklaert-3130720.html?seite=4}
\end{thebibliography}

\end{document}
